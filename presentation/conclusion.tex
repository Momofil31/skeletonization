\section{Wrap-up}
\begin{frame}[c]
  \frametitle{What we’ve learned in this lecture}
  \begin{itemize}
    \item Skeletons are useful for many image-based applications.
    \item There exist many different methods for extract a skeleton for both 2D and 3D binary images.
    \item Zhang-Suen is a fast and simple parallel thinning method which \textbf{iteratively removes pixels} from the boundaries of a region satisfying certain \textbf{conditions}.
    \item Morphological thinning iteratively remove pixels exploiting the \textbf{hit-or-miss transform} and some \textbf{ad-hoc kernels}.
    \item In morphological thinning boundary irregularities can produce a skeleton with short \textbf{spurs} that can be removed through \textbf{pruning}.
    \item Lee's 3D skeletonization method is a complex thinning algorithm which exploits \textbf{topological properties} of 3D objects.
  \end{itemize}
\end{frame}

\begin{frame}[shrink=20]
  \frametitle{Bibliography}
  \printbibliography
\end{frame}

\begin{frame}
  \frametitle{Conclusion}
  \begin{center}
    \Huge Thank you for your attention.
  \end{center}
\end{frame}